\subsection{Software Transformation Requirements} \label{sec_requirements_transformation}

In order to study stability and evolvability within cellular automata (CA) artifacts, we
will apply certain principles from the Functional-Construction software transformation, as
outlined by \textcite[251]{mannaert_normalized_2016}. The Functional Requirements
Specifications, as proposed by \textcite[254-261]{mannaert_normalized_2016}, offer a
structured framework that can be applied during the implementation phase to investigate
combinatorial effects in CA artifacts.

First, an information system must be capable of representing instances of data entities.
These data entities are composed of multiple data fields, which may either represent a
basic value or serve as a reference to another data entity. This capability ensures that
the system can efficiently store and manage complex data relationships.

Second, the system must support the execution of processing actions on instances of these
data entities. A processing action typically consists of a series of tasks, which may
either be basic units of processing, capable of independent change, or calls to other
processing actions. This ensures flexibility in how processes are managed and executed
within the system.

Third, the information system should facilitate input and output of data entity instances
through defined connectors. This requirement emphasizes the need for seamless integration
and interaction with external systems or components, allowing for data exchange in a
structured and efficient manner.

Furthermore, it is essential for an existing information system to accommodate updates to
data entities. Specifically, the system should be able to represent new versions of data
entities that include additional fields or even entirely new entities. This allows for
adaptability as the system evolves over time.

Finally, the system must be able to update its processing actions. This includes providing
new versions of processing tasks or actions, which may be mandatory for the system to use,
as well as the ability to introduce additional tasks or actions. This ensures that the
system can expand and adjust its processing capabilities in response to new requirements
or changes in the environment.

By adhering to these Functional Requirements Specifications, the system gains both the
stability to handle existing operations and the evolvability to adapt to future changes
and requirements.
