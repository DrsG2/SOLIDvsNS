\subsection{The converging principles} \label{subsec:converging_principles}

The main goal of both the \gls{srp} and \gls{soc} is to promote and encourage modularity,
low coupling, and high cohesion. While their definitions have minor nuances, the two
principles are practically interchangeable. Even though \gls{srp} does not implicitly
guarantee \gls{dvt} or \gls{avt}, it supports those theorems by directing design choices
in a certain way. One example lies in separating data models for requests, responses, and
views and respective versions of these models.

The \gls{ocp} and its relation to \gls{ns} theory emphasize the importance of designing
software entities that are open for extension but closed for modification. This principle
aligns with the \gls{ns} approach to evolvability, advocating for structures that can
adapt to new requirements without altering existing code, thus minimizing the impact of
changes. An example of this synergy can be seen in the use of expanders within \gls{ns},
which allow for introducing new functionality or data elements without disrupting the core
architecture, cohesively supporting the \gls{ocp} principle goal of extendibility and
maintainability.

The \gls{lsp} emphasizes that objects of a superclass should be replaceable with objects
of a subclass without altering the correctness of the program. This principle strongly
aligns with the emphasis on modular and replaceable components in \gls{ns}, advocating for
flexibility and the seamless integration of new functionalities. Applying this principle
within \gls{ns} is evident in designing tailored interfaces specific to a particular
version. This ensures system evolution without compromising existing functionality,
thereby upholding the \gls{lsp} directive for substitutability and system integrity.

The \gls{isp} advocates for creating specific consumer interfaces rather than one
general-purpose interface, aligning with \gls{ns} principles to enhance system
evolvability and maintainability. This alignment is evident in the modular and decoupled
design strategies advocated by both \gls{ns} and \gls{isp}, where the focus is on
minimizing unnecessary dependencies and promoting high cohesion within systems. By
applying \gls{isp}, developers can ensure that system components only depend on the
interfaces they use, which mirrors the approach in \gls{ns} to create evolvable systems by
reducing the impact of changes across modules.

The \gls{dip} and its alignment with \gls{ns} are centered on inverting the conventional
dependency structure to reduce rigidity and fragility in software systems. \gls{dip}
promotes high-level module independence from low-level modules by introducing abstractions
that both can depend on, thereby facilitating a more modular and evolvable design. This
principle mirrors the emphasis on minimizing dependencies to enhance system evolvability
in the \gls{ns} paradigm. Examples from the thesis demonstrate how leveraging \gls{dip} in
conjunction with \gls{ns} principles leads to systems that are more adaptable to change,
showcasing the practical application of these combined approaches in achieving resilient
software architectures. Designers should also be aware of the potential pitfalls of using
\gls{dip} as faulty implementations can increase combinatorial effects.

In the following table, we summarize the analysis in a tabular overview using the
following denotation:
\begin{itemize}
    \item \highlight{Strong convergence} (\fullConvergence): This indicates that the
    principles of \gls{ca} and \gls{ns} are highly converged. Both have a similar impact on
    the design and implementation.
    \item \highlight{Supports convergence} (\npartialConvergence): The \gls{ca} principle
    supports implementing the \gls{ns} principle through specific design choices. However,
    applying the \gls{ca} principle does not inherently ensure adherence to the corresponding
    \gls{ns} principle.
    \item \highlight{Weak or no convergence} (\noConvergence): The principles have no
    significant similarities in terms of their purpose, goals, or architectural supports.
\end{itemize}
    
\begin{table}[H]
    \caption{The convergence between \gls{ca} and \gls{ns} principles.}
    \renewcommand{\arraystretch}{1.5}
    \centering
    \begin{tabular}{r|llll}
    
        \textbf{\acrlong{ca}   } \textbf{   \rotatebox[origin=l]{90}{\acrlong{ns}}} & 
        \rotatebox[origin=l]{90}{\acrlong{soc}} & \rotatebox[origin=l]{90}{\acrlong{dvt}} &
        \rotatebox[origin=l]{90}{\acrlong{avt}} & \rotatebox[origin=l]{90}{\acrlong{sos}} \\
    \midrule
    
    
    \acrlong{srp} & \fullConvergence & \npartialConvergence & \npartialConvergence & \noConvergence \\
    \acrlong{ocp} & \fullConvergence & \noConvergence & \fullConvergence & \noConvergence \\
    \acrlong{lsp} & \fullConvergence & \noConvergence & \npartialConvergence & \noConvergence \\
    \acrlong{isp} & \fullConvergence & \noConvergence & \npartialConvergence & \noConvergence \\
    \acrlong{dip} & \fullConvergence & \noConvergence & \npartialConvergence & \noConvergence \\
    \bottomrule
    \end{tabular}
    \label{tab_convergence_principles_summarized}
    
\end{table}