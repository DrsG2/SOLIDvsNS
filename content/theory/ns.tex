\subsection{Normalized Systems} \label{subsec:ns}

\gls{ns} in software engineering revolves around stable and evolvable information systems,
drawing from System Theory and Statistical Entropy from Thermodynamics. \gls{ns} is rooted
in software engineering but applies to other domains, such as Enterprise Engineering
\cite{huysmans_towards_2013}, Hardware configurations like TCP-IP firewalls
\cite{haerens_evolvability_2021}, and Business Process Modeling
\cite{van_nuffel_towards_2011}.

The \gls{ns} theory emphasizes stability as a crucial property derived from the concept of
Bounded Input leading to Bounded Output (BIBO). Stability in \gls{ns} means that a bounded
functional change must result in a bounded amount of work, regardless of the system's
size. Instabilities, also referred to as combinatorial effects, occur when the number of
changes depends on the system size, negatively impacting its evolvability. 
\newline 
In the following list, we will describe the design Theorems of \gls{ns}, first presented
by \textcite{mannaert_normalized_2009}. 
\begin{itemize}
    \item \glslong{soc}: A processing function containing only a single task to achieve
    stability.
    \item \glslong{dvt}: A data structure passed through a processing function's interface
    must exhibit version transparency to achieve stability. 
    \item \glslong{avt}: A processing function that is called by another processing
    function needs to exhibit version transparency to achieve stability.
    \item \glslong{sos}: Calling a processing function within another processing function
    must exhibit state-keeping to achieve stability.
\end{itemize}

\gls{ns} aims to design evolvable software independent of the underlying technology.
Nevertheless, a particular technology must be chosen when implementing the software and
its components. For object-oriented programming languages, the following normalized
elements have been proposed \parencite[363-398]{mannaert_normalized_2016}. It is essential
to recognize that different programming languages may necessitate alternative constructs
\parencite[364]{mannaert_normalized_2016}. 

The following list describes each element using the definition from
\textcite[102]{mannaert_towards_2012}
\begin{itemize}
    \item \highlight{Data Element}: Based on \gls{dvt}, data elements have \enquote{get}
    and \enquote{set} methods for wide-sense data version transparency or marshal -and
    parse- methods for strict-sense \gls{dvt}. Supporting tasks can be added in a way
    that is consistent with the principles of \gls{soc} and \gls{dvt}.
    \item \highlight{Task Element}: Based on \gls{soc}, the core action entity can only
    contain a single functional task, not multiple tasks. Based on \gls{avt},
    arguments and parameters must be encapsulated data entities. Based on \gls{soc} and
    \gls{sos}, workflows need to be separated from action entities and will therefore be
    encapsulated in a workflow element. Based on \gls{avt}, tasks need to be encapsulated
    so that a separate action entity wraps the action entities representing task versions.
    Supporting tasks can be added in a way that is consistent with \gls{soc} and
    \gls{avt}.
    \item \highlight{Workflow Element}: Based on \gls{soc}, workflow elements cannot
    contain other functional tasks, as they are generally considered a separate change
    driver, often implemented in an external technology. Based on \gls{sos}, workflow
    elements must be stateful. This state is required for every instance of use of the
    action element and, therefore, needs to be part of, or linked to, the instance of the
    data element that serves as an argument.
    \item \highlight{Connector Element}: Based on Theorem \gls{sos}, connector elements must
    ensure that external systems can interact with data elements, but that they cannot
    call an action element in a stateless way. Supporting tasks can be added in a way that
    consistent with \gls{soc} and \gls{avt}.
    \item \highlight{Trigger Element}: Based on \gls{soc}, trigger elements need to
    control the separated —both error and non-errorstates, and check whether an action
    element has to be triggered. Supporting tasks can be added in a way that is consistent
    with \gls{soc} and \gls{avt}.
\end{itemize}