\section{conclusion}

The primary objective of \citename{koks_convergence_2023}{author} was to study the
convergence between \gls{ca} and \gls{ns} by analyzing their principles and design
elements through theory and practice. This Section will summarize the findings into a
research conclusion.

% A noteworthy distinction between \gls{ns} and \gls{ca} lies in their foundational roots.
% \gls{ns} is a product of computer science research built upon formal theories and
% principles derived from rigorous scientific investigation. Throughout this
% paper, \gls{ns} is referred to as a development approach or paradigm, it is actually a
% part of Computer Science.

Stability and evolvability are concepts not directly referenced in the literature on
\gls{ca}, but this design approach aligns with the goal of \gls{ns}. The attentive reader
can observe the shared emphasis on modularity and the separation of concerns, as all SOLID
principles strongly converge with \gls{soc}. Both approaches attempt to achieve low
coupling and high cohesion. In addition, \gls{ca} adds the dimensions of dependency
management as useful measures to improve maintainability by rigorously managing
dependencies in the Software Architecture.

The \gls{dvt} appears to be underrepresented in the SOLID principles of \gls{ca}.
\gls{dvt} is primarily supported by the \gls{srp} of \gls{ca}, as evidenced by ViewModels,
RequestModels, ResponseModels, and Entities as software elements. It is worth noting that
this application of Data Version Transparency  is an integral part of the design elements
of \gls{ca}. While \gls{ca} does address \gls{dvt} through the \gls{srp}, a more
comprehensive representation of the underlying idea of \gls{dvt} within the principles of
\gls{ca} will likely improve the convergence of \gls{ca} with \gls{ns}.

% The underrepresentation of \gls{dvt} has led to significant combinatorial effects in some
% parts of the researcher's artifacts. These combinatorial effects might be attributed to the
% author's inexperience in creating systems that enable code generation through expansion
% while maintaining stability on templates and craftings. If \gls{dvt} were better
% represented in the principles of \gls{ca}, the severity of the combinatorial effects would
% have most likely been less.

\gls{ca} Lacks a strong foundation for receiving external triggers in its design
philosophy. This is partially represented by the Controller element. However, this element
is described as being used for web-enabled environments and might result in a less
comprehensive approach to receiving external triggers across various technologies or
systems.

The most notable difference between \gls{ca} and \gls{ns} is their approach to handling
state. \gls{ca} does not explicitly address state management in its principles or design
elements. \gls{ns} Provides the principle of \gls{sos}, ensuring that state changes within
a software system are stable and evolvable. This principle can be crucial in developing
scalable and high-performance systems, as it isolates state changes from the rest of the
system, reducing the impact of state-related dependencies and side effects. 

The findings can only lead to the conclusion that the convergence between \gls{ca} and
\gls{ns} is incomplete.  Consequently, \gls{ca} cannot fully ensure stable and evolvable
software artifacts as \gls{ns} has defined them.

While it has been demonstrated that the convergence between these two approaches is
incomplete, combining both methodologies is highly beneficial for \gls{ns} and \gls{ca}
for various reasons. The primary advantage of synergizing them lies in the complementary
nature of both paradigms, where each approach provides strengths that can be leveraged to
address a robust architectural design. 

\gls{ca} offers a well-defined, practical, and modular structure for software development.
Its principles, such as SOLID, guide developers in creating maintainable, testable, and
scalable systems. This architectural design approach is highly suitable for various
applications and can be easily integrated with the theoretical foundations provided by
\gls{ns}. Conversely, the \gls{ns} approach offers a more comprehensive theoretical
understanding of achieving stable and evolvable systems. 

To conclude, the popularity and widespread adoption of \gls{ca} in the software
development community can benefit \gls{ns}. As more developers adopt \gls{ca}, they become
more familiar with \gls{ns} and recognize their value to software design. Synergizing both
approaches will likely lead to increased adoption of \gls{ns}.